\documentclass{article}
\usepackage{amsmath}
\usepackage{hyperref}
\usepackage{graphicx}
%\usepackage[hidelinks]{hyperref}
\textwidth 6.6in \addtolength{\oddsidemargin}{-2.2cm} \textheight 9in \addtolength{\topmargin}{-1in} \setlength{\parindent}{0pt} \setlength{\parskip}{0.5cm} \topskip 0.0in \everymath{\displaystyle} \pagestyle{empty}
\title{Introduction to Software Engineering \\ Easy Cloud}
\author{Rishabh Nigam(10598)  Pankaj Prateek(10497)}
\date{Nov 18, 2013}

\hypersetup{
    colorlinks=false,
    pdfborder={0 0 0},
}
\begin{document}
\maketitle

\section{Introduction}
Easy Cloud serves as a replacement to the paid services provided by Cloud Storage Services such as Dropbox, Google and Skydrive. Through this system, we aim to provide the user an abstract layer over various cloud services, which would give the impression of a larger amount of data being accessible on the cloud by combining the spaces provided by these services.

Though there are various paid services which provide a larger amount of cloud space,  but they charge too much. On the other hand, people have free space available at various cloud services. The initial motivation was to use this space. The idea is to make it in a more robust and scalable way, providing the user a drive like structure (like Dropbox and Google Drive desktop clients). 

EasyCloud provides you a drive on your local disk which is synced to one/more of the online storage sites. The files are currently synced to Dropbox, Google Drive and Skydrive.  The target audience for this project includes all the people who are using cloud storage services.

Currently, the drive has been implemented and throughly tested for linux. We plan to extend the support to windows in upcoming releases.

\section{Functionality}
The app on installation asks the user to create a folder named "easyCloud" and put the files it wants to sync in this folder only. There needs to be a same named folder in the cloud locations in Dropbox, Google Drive, and Skydrive. The easyCloud folder on the local drive is synced with the combination of corresponding folders on \{Dropbox, Google Drive, and Skydrive\}.

We have a basic GUI which allows user to choose some set of sync settings which are as follows.
\begin{enumerate}
	\item Upload Location. The following options are available.
	\begin{itemize}
	\item Split: It stores the file in exactly one of the place on the cloud. This is decided by an algorithm which is described in later sections.
	\item All: This simultaneously updates the file to all the places on the cloud which are authenticated by the user
	\item Only to Dropbox.
	\item Only to SkyDrive.
	\item Only to GoogleDrive.
	\end{itemize}
	\item Sync:
	\begin{itemize}
	\item A sync button is provided on clicking which the files in the local drive are synced to the files on the cloud.
	\item A sync is also carried out every 15 minutes.
	\item Event driven sync is also implemented. In this case, if the application detects a change in the file structure of the easyCloud folder on the local drive (by comparing it to an older version of the list every 2 seconds), the files are synced. Such event driven sync can also be implemented for cloud services, but getting the file lists over the cloud takes a considerable amount of time causing the application to become inresponsive.
	\end{itemize}
	\item Authentication: There are 3 buttons provided in the GUI to authenticate each of Dropbox, GoogleDrive and SkyDrive repectively. On clicking those button, the user is authenticated using the stored authentication tokens. If the tokens are not found or have expired, instructions are given to the user to authenticate. The user is provided with a authentication link, which needs to be opened in the browser and the code returned in this process, works as an authentication token for the user.
\end{enumerate}

\section{Implementation}
The complete implementation is done in python 2.7. {\em python-tkinter} is used for GUI. The API's of various services are taken from their developer websites.

\begin{enumerate}
\item A folder named "easyCloud" is created in the home folder of the user which is synced with folder of same name in the home directories of each of the linked accounts.
\item If a file is not present in any of the cloud services, but is found in the local folder, it is uploaded to the cloud. The location is decided by the location algorithm.
\item If a file is present in some cloud service but was not present in the local folder previously, it is downloaded to the local folder. This would happen when a user directly uploads the file in the cloud service (using web/other services).
\item But if a file is present in some cloud and was also present in the local folder previously, it is deleted. This would happen when a user deletes a synced file from the "easyCloud" folder.
\item {\bf Location Algorithm}: The user is presented with 5 choices, "Split", "All", "Only To Dropbox", "Only To skyDrive" and "Only To Google Drive". When any of the last four is selected, the file is uploaded accordingly to the selected choice. But if "Split" option is selected, the application checks which of the linked accounts has the highest (free space/total space) ratio and the file is uploaded in the corresponding account.
\item {\bf Sync}: There are two sync options, in the first, the folders will automatically sync every 15 minutes, and the other is event driven, i.e., when a user uplaods or deletes a files from the local folder, the folders are synced. Apart from this, the user is also presented with a button in the GUI which would sync the folders. 
\end{enumerate}

\section{Testing}
The application has been extensively tested on various linux machines. Various combination of tasks like upload, download, delete using various sync settings have been tested.
Some bugs and inconsistencies were found in the testing processs. These were corrected followed by a subsequent phase of testing.

\section{Furthur Work}
\begin{enumerate}
\item Port the application to Windows and other operating systems.
\item Include other services like box.net, copy.com, amazon cloud etc. 
\item Develop and implemenet better algorithms for dividing files over various cloud services.
\end{enumerate}

\section{References}
\begin{itemize}
\item Dropbox: https://www.dropbox.com/developers
\item Google Drive: https://developers.google.com/drive/ 
\item SkyDrive: http://msdn.microsoft.com/en-us/library/live/hh826521.aspx 
\item The project had been hosted on Github at https://github.com/pankajprateek/easyCloud
\end{itemize}

\end{document}

Software Requirements
Specification
For
Centralized Cloud Storage
Version 1.0
Prepared by
Rishabh Nigam
Pankaj Prateek
IIT Kanpur
August 22, 2013
Software Requirements Specification for Centralized Cloud Storage
 Page ii
Table of Contents
Table of Contents ..................................................................................................................... ii
Revision History ....................................................................................................................... ii
1.Introduction ......................................................................................................................... 1
1.1
 Purpose....................................................................................................................................... 1
1.2
 Document Conventions ............................................................................................................... 1
1.3
 Intended Audience ...................................................................................................................... 1
1.4
 Product Scope ............................................................................................................................. 1
1.5
 References .................................................................................................................................. 1
2. Overall Description ............................................................................................................. 2
2.1
 Product Perspective..................................................................................................................... 2
2.2
 Product Functions ....................................................................................................................... 2
2.3
 User Classes and Characteristics ................................................................................................. 3
2.4
 Operating Environment ............................................................................................................... 3
2.5
 Design and Implementation Constraints ...................................................................................... 3
2.6
 User Documentation ................................................................................................................... 3
2.7
 Assumptions and Dependencies .................................................................................................. 3
3. External Interface Requirements ....................................................................................... 4
3.1 User Interfaces ............................................................................................................................ 4
3.2 Communications Interfaces ......................................................................................................... 4
3.3 Software Interfaces ..................................................................................................................... 4
4. System Features ................................................................................................................... 4
4.1 General Use Cases ...................................................................................................................... 4
4.2 Managing Data ........................................................................................................................... 4
5. Other Nonfunctional Requirements ................................................................................... 5
5.1 Performance Requirements ......................................................................................................... 5
5.2 Security Requirements ................................................................................................................ 5
Revision History
Name
 Date
 Reason For Changes
Pankaj Prateek
 21/08/13
 Created the SRS Document
Version
1.0
Software Requirements Specification for Centralized Cloud Storage
 Page 1
1. Introduction
1.1 Purpose
This document provides the software requirements of the Centralized Cloud Storage (CCS). It also
explains the detailed functionality to the end users. Pre-requisites for getting the maximum of this
document are being familiar with Cloud Storage Services such as Dropbox, Drive, box.net.
1.2 Document Conventions
The following Acronyms have been used in this document.
 CCS: Centralized Cloud Storage.
 CSS: Cloud Storage Service
Whenever we refer cloud storage services, we are talking about any general element from the
set {Dropbox, Google-Drive, box.net, etc}
1.3 Intended Audience
The target audience of this app is the end users and developers who want to extend the system, or
build their own personalized versions of the CCS.
1.4 Product Scope
CCS serves as a replacement to the paid services provided by a CSS. We though this framework
want to provide a larger amount of data being accessible on the cloud. The initial motive to work on
this problem has been the work of some of the people implementing hacky prototype of the model.
But as there isn’t a system which is in use as of now, the idea is to make it in a more robust and
scalable way.
1.5 References
1.API’s
 
 Box.net:
http://developers.box.net/w/page/12923958/Welcome%20to%20the%20Box%20Platfor
m
2.3.4.https://developer.mozilla.org/en-US/
http://www.w3schools.com/
http://php.net/
Software Requirements Specification for Centralized Cloud Storage
 Page 2
2. Overall Description
2.1 Product Perspective
This project is a first time project in this field. There have been some earlier attempts in hack-a-
thons, we are trying to build a commercial project based on the idea. The target audience for this
project includes all the people who are using cloud storage services at this point of time (which is a
huge chunk of people, as of Nov. 2012, Dropbox had 100 million Daily Active Users). As it’s a
mobile site, it could be used on Chrome/Firefox/Safari which is where the majority of people use
internet from.
2.2 Product Functions
2.3 User Classes and Characteristics
There is only one type of user. There are no pre-requisites, anyone who uses cloud storage services
such as Dropbox, Google drive, etc. can use the site.
2.4 Operating Environment




System will be deployed on a central server which would have a HTTP Server, an
Application Manager such as Tomcat, a FTP server and a Database Server.
Users would use the system through a HTTP browser that permits cookies to be stored.
The HTTP Server used would be Apache on Linux.
The database server used would by MySQL.
2.5 Design and Implementation Constraints
The server should have sufficient resources (CPU, RAM, and Disk Space) to service the required
number of users.
 3 GHz 8 GB RAM PC, 100 GB HDD to support around 1000 active users.
 APIs of various cloud storage services would be used, which would be in Python (2.7.3),
PHP and JavaScript.
 The system would be available only in English
 The system would have a web interface.
 Data is stored only temporarily on the server so there is no security threat on the data.
 The server must support Python 2.7.3.
 The SDKs for various cloud hosting services should be supported by the server.
2.6 User Documentation
A documentation of the application would be present on the application website for the user.
The UI of the application would be kept simple so as to allow users to use it easily without much
dependence on the documentation.
2.7 Assumptions and Dependencies
For full working of the system the computers that the users use to connect to the system should be
able to access the server on which the system is deployed.
 The server should remain powered on as far as possible and in case of a power failure, there
should be indefinite power back up or a controlled shutdown should be performed.
 The browsers used to access the system should support cookies.
Software Requirements Specification for Centralized Cloud Storage
 Page 4
3. External Interface Requirements
3.1 User Interfaces
The User Interface is built in PHP, HTML5, CSS and JavaScript. UI design is an important of our
system, so various guidelines are strictly adhered to.
3.2 Communication Interfaces
.



HTTP could be used to communicate between the server hosting the application and the
client browser.
FTP/SFTP will be used for data transfer.
SMTP server will be used for sending mails
3.3 Software Interfaces
There are five servers
 HTTP server
 Application server
 Database server. MySQL would be used
 FTP server.
 SMTP server.
4. System Features
4.1 General Use Cases:
4.1.0
Login
The user enters a login and password and clicks Login. The user is logged in if the
username, password combination corresponds to a valid user of the system.
If the user is already logged in, the logged in data is fetched from the cookie on the
browser, and the user connection is maintained. During the logout this cookie is cleared.
4.1.1
Change Password
Pre-Condition: User is logged in.
1.2.3.A link to change password is provided in the account info section of the application site.
The user needs to enter the old password and the new password twice
The password is then changed.
4.1.2
 Forgot Password
1.2.3.User clicks forgot password, enters his username/email-address.
The password (or the link to reset the password) is sent to the mailing address
A message is displayed that “the password was sent by mail”
Software Requirements Specification for Centralized Cloud Storage
 Page 5
4.2 Managing Data:
4.2.1 Linking the account on the application website with the accounts on various
cloud storage service.
 Authentication by the user for accounts on various Cloud Storage Services. This step
involves the user allowing the application website access to modify/read the data
stored on these cloud services.
 Generation of authentication token for the cloud storage service.
 Storing the authentication token on the application website SQL servers for future
use.
4.2.2 Manage the files already on the cloud storage.
 After the account is linked, the file-list on the cloud services for the linked accounts
is fetched and displayed to the user.
 The file is fetched only when the user wants to download the file.
4.2.3 Add/remove/move files.
 The user can add/remove the files on the application website. It can also use move
the file from one storage system to another storage service.
5. Other Nonfunctional Requirements
5.1 Performance Requirements



The server should have a fast CPU, hard disk, and RAM (e.g. 3Ghz, 100 GB SCSI, 8 GB
RAM)
The system should provide quick response time
The system should be able to serve at least 1000 users simultaneously.
5.2 Security Requirements


Users should not have access to other users’ data and should not be able to perform
administrative tasks.
The “forgot password” mechanism should mail the password to the user rather than asking
for the answer to a secret question.
